%%%%%%%%%%%%%%%%%%%%%%%%%%%%%%%%%%%%%%%%%%%%%%%%%%%%%%%%
%%%%%%%%%%%%%%%%%%%%%%%%%%%%%%%%%%%%%%%%%%%%%%%%%%%%%%%%
%%%%%%%%%%%%%%%%%%%%%%%%%%%%%%%%%%%%%%%%%%%%%%%%%%%%%%%%
%%%%%%%%%%%%%%%%%%%%%%%%%%%%%%%%%%%%%%%%%%%%%%%%%%%%%%%%
\documentclass[a4paper,12pt,twoside,openright]{report}
\usepackage[bibjustif]{abntex2cite}
\usepackage[brazil]{babel}
\usepackage{setspace}
\usepackage{amsthm}
\usepackage[figuresright]{rotating}
\usepackage{graphics}
\usepackage{amssymb}
\usepackage{graphicx}
\usepackage{fancybox}
\usepackage{amsmath}
\usepackage{picinpar}
\usepackage{colortbl}
\usepackage{wasysym}
\usepackage{txfonts}
\usepackage{pb-diagram}
\usepackage{relsize}
\usepackage{tikz}
\usepackage{pgfplots}
\usepackage{subfigure}
\usepackage{algorithm}
\usepackage{algorithmic}
\usepackage[hang,small,bf]{caption}
\usepackage[compact]{titlesec}
\usepackage[left=3cm,top=3cm,right=2cm,bottom=3cm]{geometry}%esq e sup 3cm  ; dir e inf 2cm
\usepackage[subfigure]{tocloft}
\usepackage{subfigure}

\renewcommand{\cftfigfont}{Figura }
\renewcommand{\cfttabfont}{Tabela }

\renewcommand{\cftfigaftersnum}{ - }
\renewcommand{\cfttabaftersnum}{ - }

\captionsetup{font=footnotesize,labelsep=endash}

\def\baselinestretch{1.6}
\pagestyle{myheadings}

\begin{document}
\titlespacing{\section}{0cm}{1.5cm}{1.5cm}
\titlespacing{\subsection}{0cm}{1.5cm}{1.5cm}

%%%%%%%%%%%%%%%%%%%%%%%%%%%%%%%%%%%%%%%%%%%%%%%%%%%%%%%%
%%%%%%%%%%%%%%%%%%%%%%%%%%%%%%%%%%%%%%%%%%%%%%%%%%%%%%%%
% CAPA
%%%%%%%%%%%%%%%%%%%%%%%%%%%%%%%%%%%%%%%%%%%%%%%%%%%%%%%%
%%%%%%%%%%%%%%%%%%%%%%%%%%%%%%%%%%%%%%%%%%%%%%%%%%%%%%%%
\setboolean{@twoside}{true}
\title{
	\vspace*{-120pt}
	\hspace*{+55pt}Universidade Estadual Paulista
	\newline
	\hspace*{+45pt}Instituto de Bioci\^{e}ncias, Letras e Ci\^{e}ncias Exatas
	\newline
	\hspace*{+30pt}Departamento de Ci\^{e}ncia da Computa\c{c}\~{a}o e Estat\'{i}stica
	\newline
	\newline
	\newline
	\hspace*{+50pt}Luis Fernando Teixeira Silva
	\newline
	\newline
	\newline
	Um sistema para reconhecimento de comandos falados dependente do locutor
}

\author{{\tiny .}}
\date{\vspace*{+60pt}S\~{a}o Jos\'{e} do Rio Preto - SP \\ 2017}
\maketitle

\newpage\
\thispagestyle{empty}

%%%%%%%%%%%%%%%%%%%%%%%%%%%%%%%%%%%%%%%%%%%%%%%%%%%%%%%%
%%%%%%%%%%%%%%%%%%%%%%%%%%%%%%%%%%%%%%%%%%%%%%%%%%%%%%%%
% PAGINA DE ROSTO
%%%%%%%%%%%%%%%%%%%%%%%%%%%%%%%%%%%%%%%%%%%%%%%%%%%%%%%%
%%%%%%%%%%%%%%%%%%%%%%%%%%%%%%%%%%%%%%%%%%%%%%%%%%%%%%%%
\pagestyle{empty}
\begin{tabular}{p{430pt}}
\pagestyle{empty}
\begin{center}
\pagestyle{empty}
\LARGE{\hspace*{+50pt}Luis Fernando Teixeira Silva\newline \newline \newline \newline \newline Um sistema para reconhecimento de comandos falados dependente do locutor}
\end{center}

\vspace*{+30pt}
\hspace*{+200pt}
\begin{tabular}{p{200pt}}
	\singlespacing{\small Monografia apresentada ao Programa de gradua\c{c}\~{a}o em Ci\^{e}ncia da Computa\c{c}\~{a}o da UNESP para obten\c{c}\~{a}o do t\'{i}tulo de Bacharel. \newline \newline Orientador: Prof. Dr. Rodrigo Capobianco Guido} \\
\end{tabular}
\newline 
\newline
\begin{center}
{\large S\~{a}o Jos\'{e} do Rio Preto - SP \\ 2017}
\end{center}
\end{tabular}

%%%%%%%%%%%%%%%%%%%%%%%%%%%%%%%%%%%%%%%%%%%%%%%%%%%%%%%%
%%%%%%%%%%%%%%%%%%%%%%%%%%%%%%%%%%%%%%%%%%%%%%%%%%%%%%%%
% FICHA CATALOGRAFICA
%%%%%%%%%%%%%%%%%%%%%%%%%%%%%%%%%%%%%%%%%%%%%%%%%%%%%%%%
%%%%%%%%%%%%%%%%%%%%%%%%%%%%%%%%%%%%%%%%%%%%%%%%%%%%%%%%
\newpage
\thispagestyle{empty}
\noindent
AUTORIZO A DIVULGA\c{C}\~{A}O TOTAL OU PARCIAL DESTE TRABALHO, POR\\QUALQUER MEIO CONVENCIONAL OU ELETR\^{O}NICO, PARA  FINS DE ES-\\TUDO E PESQUISA, DESDE QUE CITADA A FONTE

\begin{center}
\vspace*{+260pt}
{\footnotesize \hspace*{-81pt}Ficha catalogr\'{a}fica elaborada pelo Servi\c{c}o de Biblioteca do IBILCE/UNESP}
\end{center}

\vspace*{+5pt}

\begin{tabular}{|p{12.5cm}|}
\hline
{\singlespacing
\vspace*{-20pt}
\hspace*{+40pt} {\small Luis Fernando Teixeira Silva} \newline
\hspace*{+40pt} {\small titulo} \newline
\hspace*{+20pt} {\small titulo} \newline
\hspace*{+20pt} {\small titulo. / fulano de tal; orientador} \newline
\hspace*{+20pt} {\small Rodrigo Capobianco Guido. S\~{a}o Jos\'{e} do Rio Preto, 2017.} \newline
\hspace*{+40pt} {\small xxx p.} \newline

\hspace*{+40pt} {\small Monografia (TCC} \newline
\hspace*{+20pt} {\small TCC} \newline
\hspace*{+20pt} {\small TCC, 2017.} \newline \newline

\hspace*{+40pt} {\small 1. Processamento de sinais. 2. Reconhecimento de locutor. 3. Ac\'{u}s-} \newline
\hspace*{+20pt} {\small tica. 4. Energia. 5. Escala \textit{Bark}. I. Capobianco Guido, Rodrigo, orient.} \newline
\hspace*{+20pt} {\small II. T\'{i}tulo.} \newline \newline
}
\\
\hline
\end{tabular}

%%%%%%%%%%%%%%%%%%%%%%%%%%%%%%%%%%%%%%%%%%%%%%%%%%%%%%%%
%%%%%%%%%%%%%%%%%%%%%%%%%%%%%%%%%%%%%%%%%%%%%%%%%%%%%%%%
% FOLHA DE APROVACAO: MANTER EM BRANCO
%%%%%%%%%%%%%%%%%%%%%%%%%%%%%%%%%%%%%%%%%%%%%%%%%%%%%%%%
%%%%%%%%%%%%%%%%%%%%%%%%%%%%%%%%%%%%%%%%%%%%%%%%%%%%%%%%
\thispagestyle{empty}
\newpage\ \thispagestyle{empty} \newpage
\thispagestyle{empty}
\newpage\ \thispagestyle{empty} \newpage
\thispagestyle{empty}

%%%%%%%%%%%%%%%%%%%%%%%%%%%%%%%%%%%%%%%%%%%%%%%%%%%%%%%%
%%%%%%%%%%%%%%%%%%%%%%%%%%%%%%%%%%%%%%%%%%%%%%%%%%%%%%%%
% DEDICATORIA
%%%%%%%%%%%%%%%%%%%%%%%%%%%%%%%%%%%%%%%%%%%%%%%%%%%%%%%%
%%%%%%%%%%%%%%%%%%%%%%%%%%%%%%%%%%%%%%%%%%%%%%%%%%%%%%%%
\newpage
\thispagestyle{empty}
\vspace*{+480pt}
\hspace*{+160pt}
\begin{tabular}{p{200pt}}
	{\small Dedico ao...}
\end{tabular}

\newpage\ \thispagestyle{empty} \newpage

%%%%%%%%%%%%%%%%%%%%%%%%%%%%%%%%%%%%%%%%%%%%%%%%%%%%%%%%
%%%%%%%%%%%%%%%%%%%%%%%%%%%%%%%%%%%%%%%%%%%%%%%%%%%%%%%%
% AGRADECIMENTOS
%%%%%%%%%%%%%%%%%%%%%%%%%%%%%%%%%%%%%%%%%%%%%%%%%%%%%%%%
%%%%%%%%%%%%%%%%%%%%%%%%%%%%%%%%%%%%%%%%%%%%%%%%%%%%%%%%
\newpage
\thispagestyle{empty}
\begin{center}{\LARGE \textbf{Agradecimentos}}\end{center}
\vspace{+60pt}
A Deus, ....

\newpage\ \thispagestyle{empty} \newpage

%%%%%%%%%%%%%%%%%%%%%%%%%%%%%%%%%%%%%%%%%%%%%%%%%%%%%%%%
%%%%%%%%%%%%%%%%%%%%%%%%%%%%%%%%%%%%%%%%%%%%%%%%%%%%%%%%
% PENSAMENTO
%%%%%%%%%%%%%%%%%%%%%%%%%%%%%%%%%%%%%%%%%%%%%%%%%%%%%%%%
%%%%%%%%%%%%%%%%%%%%%%%%%%%%%%%%%%%%%%%%%%%%%%%%%%%%%%%%
\newpage
\vspace*{+480pt}
\thispagestyle{empty}
\begin{flushright}
\textit{``No fim tudo d\'{a} certo, e se n\~{a}o deu certo \'{e} porque ainda n\~{a}o chegou ao fim.''}
\\
\textbf{Fernando Sabino}
\end{flushright}

\newpage\ \thispagestyle{empty} \newpage

%%%%%%%%%%%%%%%%%%%%%%%%%%%%%%%%%%%%%%%%%%%%%%%%%%%%%%%%
%%%%%%%%%%%%%%%%%%%%%%%%%%%%%%%%%%%%%%%%%%%%%%%%%%%%%%%%
% RESUMO E ABSTRACT
%%%%%%%%%%%%%%%%%%%%%%%%%%%%%%%%%%%%%%%%%%%%%%%%%%%%%%%%
%%%%%%%%%%%%%%%%%%%%%%%%%%%%%%%%%%%%%%%%%%%%%%%%%%%%%%%%
\newpage
\thispagestyle{empty}
\noindent
\begin{center}\textbf{\huge{Resumo}}\end{center} 
\vspace*{+60pt}
\begin{singlespace}
TAL, F. \textit{titulo}. 2016. xxxp. TCC UNESP 2016.
\end{singlespace}
\vspace*{+20pt}
Atualmente, ....
\\
\\
Palavras-chave: Processamento de sinais. Reconhecimento de locutor. Ac\'{u}stica. Escala \textit{Bark}.

\newpage\ \thispagestyle{empty} \newpage

\newpage
\thispagestyle{empty}
\noindent
\begin{center}\textbf{\huge{Abstract}}\end{center} 
\vspace*{+60pt}
\begin{singlespace}
TAL, F. \textit{titulo}. 2016. xxxp. TCC UNESP 2016.
\end{singlespace}
\vspace*{+20pt}
Nowadays, ...
\\
\\
Keywords: Signal processing. Speaker recognition. Acoustics. Bark scale.

\newpage\ \thispagestyle{empty} \newpage

%%%%%%%%%%%%%%%%%%%%%%%%%%%%%%%%%%%%%%%%%%%%%%%%%%%%%%%%
%%%%%%%%%%%%%%%%%%%%%%%%%%%%%%%%%%%%%%%%%%%%%%%%%%%%%%%%
% INDICES
%%%%%%%%%%%%%%%%%%%%%%%%%%%%%%%%%%%%%%%%%%%%%%%%%%%%%%%%
%%%%%%%%%%%%%%%%%%%%%%%%%%%%%%%%%%%%%%%%%%%%%%%%%%%%%%%%
%\pagestyle{empty}
\renewcommand
\listfigurename{\hspace*{+120pt} Lista de Figuras \thispagestyle{empty}}
\pagestyle{empty}
\listoffigures
%\pagestyle{empty}
\newpage\ \thispagestyle{empty} \newpage
%\pagestyle{empty}
\renewcommand
\listtablename{\hspace*{+120pt} Lista de Tabelas \thispagestyle{empty}}
%\pagestyle{empty}
\listoftables
%\pagestyle{empty}

\newpage\ \thispagestyle{empty} \newpage

%%%%%%%%%%%%%%%%%%%%%%%%%%%%%%%%%%%%%%%%%%%%%%%%%%%%%%%%
%%%%%%%%%%%%%%%%%%%%%%%%%%%%%%%%%%%%%%%%%%%%%%%%%%%%%%%%
% LISTA DE ABREVIACOES
%%%%%%%%%%%%%%%%%%%%%%%%%%%%%%%%%%%%%%%%%%%%%%%%%%%%%%%%
%%%%%%%%%%%%%%%%%%%%%%%%%%%%%%%%%%%%%%%%%%%%%%%%%%%%%%%%
\newpage
\thispagestyle{empty}
\vspace*{+62pt}
\begin{center}{\huge \textbf{Lista de Abreviaturas}}\end{center}
\vspace*{+20pt}
\begin{tabular}{l l}
\textbf{ABFF}&Academia Brasileira de Fonoaudiologia Forense
\\
\textbf{CHMM}&\textit{Continuous Hidden Markov Models}
\\
\textbf{DTW}&\textit{Dynamic Time Warping}
\\
\textbf{DWT}&\textit{Discrete Wavelet Transform}
\\
\textbf{EER}&\textit{Equal Error Rate}
\\
\textbf{FIR}&\textit{Finite Impulse Response}
\\
\textbf{GMM-UBM}&\textit{Gaussian Mixture Models-Universal Backgound Models}
\\
\textbf{HMM}&\textit{Hidden Markov Model}
\\
\textbf{IEEE}&\textit{Institute of Electrical and Eletronic Engineers}
\\
\textbf{IIR}&\textit{Infinite Impulse Response}
\\
\textbf{LPCC}&\textit{Linear Predictive Cepstrum Coefficients}
\\
\textbf{MFCC}&\textit{Mel Frequency Cepstral Coefficients}
\\
\textbf{MLP}&\textit{Multiple Layer Perceptron}
\\
\textbf{NIST}&\textit{National Institute of Standards and Technology}
\\
\textbf{PCM}&\textit{Pulse Code Modulation}
\\
\textbf{RNA}&Rede Neural Artificial
\\
\textbf{SVM}&\textit{Support Vector Machine}
\\
\textbf{TIMIT}&\textit{Texas Instruments and Massachusetts}
\\
\textbf{TTVNN}&\textit{Transi\c{c}\~{a}o entre Trechos Vozeados e N\~{a}o-Vozeados}
\\
\textbf{VQ}&\textit{Vector Quantization}
\\
\textbf{WAVE}&\textit{Waveform}
\\
\textbf{WPT}&\textit{Wavelet Packet Transform}
\\
\textbf{ZCR}&\textit{Zero Crossing Rate}
\end{tabular}

%%%%%%%%%%%%%%%%%%%%%%%%%%%%%%%%%%%%%%%%%%%%%%%%%%%%%%%%
%%%%%%%%%%%%%%%%%%%%%%%%%%%%%%%%%%%%%%%%%%%%%%%%%%%%%%%%
% SUMARIO
%%%%%%%%%%%%%%%%%%%%%%%%%%%%%%%%%%%%%%%%%%%%%%%%%%%%%%%%
%%%%%%%%%%%%%%%%%%%%%%%%%%%%%%%%%%%%%%%%%%%%%%%%%%%%%%%%
\newpage\ \thispagestyle{empty} \newpage
%\pagestyle{empty}
\renewcommand
\contentsname{\hspace*{+160pt} Sum\'{a}rio \thispagestyle{empty}}
%\pagestyle{empty}
\tableofcontents
%\pagestyle{empty}

%%%%%%%%%%%%%%%%%%%%%%%%%%%%%%%%%%%%%%%%%%%%%%%%%%%%%%%%
%%%%%%%%%%%%%%%%%%%%%%%%%%%%%%%%%%%%%%%%%%%%%%%%%%%%%%%%
% CAPITULO 1
%%%%%%%%%%%%%%%%%%%%%%%%%%%%%%%%%%%%%%%%%%%%%%%%%%%%%%%%
%%%%%%%%%%%%%%%%%%%%%%%%%%%%%%%%%%%%%%%%%%%%%%%%%%%%%%%%
\setboolean{@twoside}{true}
\chapter{Introdu\c{c}\~{a}o}
\thispagestyle{myheadings}
\pagestyle{myheadings}
%%%%%%%%%%%%%%%%%%%%%%%%%%%%%%%%%%%%%%%%%%%%%%%%%%%%%%%%
% SECAO
%%%%%%%%%%%%%%%%%%%%%%%%%%%%%%%%%%%%%%%%%%%%%%%%%%%%%%%%
\section{Considera\c{c}\~{o}es iniciais}
\label{cap1}
\par Petry \cite{PETRY2002} define o reconhecimento de locutores como...
%%%%%%%%%%%%%%%%%%%%%%%%%%%%%%%%%%%%%%%%%%%%%%%%%%%%%%%%
% SECAO
%%%%%%%%%%%%%%%%%%%%%%%%%%%%%%%%%%%%%%%%%%%%%%%%%%%%%%%%
\section{Objetivos}
\par Este trabalho tem como objetivo implementar um algoritmo computacional desenvolvido em C para reconhecer comandos falados dependente do locutor.
%%%%%%%%%%%%%%%%%%%%%%%%%%%%%%%%%%%%%%%%%%%%%%%%%%%%%%%%
% SECAO
%%%%%%%%%%%%%%%%%%%%%%%%%%%%%%%%%%%%%%%%%%%%%%%%%%%%%%%%
\section{Metodologia}
\par Para a elabora\c{c}\~{a}o deste projeto foi determinado os seguintes 11 comandos:
\begin{itemize}
	\item{}Bom dia, Logan;
	\item{}Bom noite, Logan;
	\item{}Oi, Logan;
	\item{}Como est\'{a} o tempo hoje?;
	\item{}vai chover?;
	\item{}Abrir calculadora;
	\item{}Ver not\'{i}cias;
	\item{}Pesquisar;
	\item{}Alarme;
	\item{}Calend\'{a}rio;
	\item{}Sair;	
\end{itemize}sendo que posteriormente foi realizada a grava\c{c}\~{a}o de 10 \'{a}udios para cada um dos 11 comandos referidos, totalizando 110 arquivos de \'{a}udio no formato \textit{wave}.
%%%%%%%%%%%%%%%%%%%%%%%%%%%%%%%%%%%%%%%%%%%%%%%%%%%%%%%%
% SECAO
%%%%%%%%%%%%%%%%%%%%%%%%%%%%%%%%%%%%%%%%%%%%%%%%%%%%%%%%
\section{Organiza\c{c}\~{a}o do trabalho}
\par O texto vindouro do presente trabalho est\'{a} organizado da seguinte forma:
\begin{itemize}
\item{}No Cap\'{i}tulo \ref{cap2} apresenta-se uma s\'{e}rie de trabalhados publicados envolvendo a \'{a}rea de reconhecimento de locutores, mostrando como s\~{a}o in\'{u}meras as possibilidades de se realizar essa tarefa. Exp\~{o}e-se, tamb\'{e}m, os principais conceitos e teorias que est\~{a}o relacionados com o trabalho que foi desenvolvido.
\item{}No Cap\'{i}tulo \ref{cap3} apresenta-se, com detalhes, todo o desenvolvimento do trabalho proposto e de que forma os conceitos discutidos no cap\'{i}tulo anterior foram utilizados.
\item{}No Cap\'{i}tulo \ref{cap4} relatam-se todos os resultados obtidos no trabalho, a partir dos testes de reconhecimento que foram realizados.
\item{}No Cap\'{i}tulo \ref{cap5} apresentam-se as conclus\~{o}es sobre o trabalho, bem como propostas para pesquisas futuras.
\end{itemize}
%%%%%%%%%%%%%%%%%%%%%%%%%%%%%%%%%%%%%%%%%%%%%%%%%%%%%%%%
%%%%%%%%%%%%%%%%%%%%%%%%%%%%%%%%%%%%%%%%%%%%%%%%%%%%%%%%
% CAPITULO 2
%%%%%%%%%%%%%%%%%%%%%%%%%%%%%%%%%%%%%%%%%%%%%%%%%%%%%%%%
%%%%%%%%%%%%%%%%%%%%%%%%%%%%%%%%%%%%%%%%%%%%%%%%%%%%%%%%
\chapter{Revis\~{a}o Bibliogr\'{a}fica}
\label{cap2}
\thispagestyle{myheadings}
%%%%%%%%%%%%%%%%%%%%%%%%%%%%%%%%%%%%%%%%%%%%%%%%%%%%%%%%
% SECAO
%%%%%%%%%%%%%%%%%%%%%%%%%%%%%%%%%%%%%%%%%%%%%%%%%%%%%%%%
\section{Fundamenta\c{c}\~{a}o da Verifica\c{c}\~{a}o de Locutores}
\label{secao_reconhecimento}
\par Reconhecimento de locutores \cite{BEIGI2011}...
%%%%%%%%%%%%%%%%%%%%%%%%%%%%%%%%%%%%%%%%%%%%%%%%%%%%%%%%
% SECAO
%%%%%%%%%%%%%%%%%%%%%%%%%%%%%%%%%%%%%%%%%%%%%%%%%%%%%%%%
\section{Arquivos Ac\'{u}sticos no Formato \textit{WAVE}}
\label{secao_formato wave}
\par O formato...
%%%%%%%%%%%%%%%%%%%%%%%%%%%%%%%%%%%%%%%%%%%%%%%%%%%%%%%%
% SECAO
%%%%%%%%%%%%%%%%%%%%%%%%%%%%%%%%%%%%%%%%%%%%%%%%%%%%%%%%
\section{Energia}
\label{energia}
\par A energia...

%%%%%%%%%%%%%%%%%%%%%%%%%%%%%%%%%%%%%%%%%%%%%%%%%%%%%%%%
% SECAO
%%%%%%%%%%%%%%%%%%%%%%%%%%%%%%%%%%%%%%%%%%%%%%%%%%%%%%%%
\section{Vetores de Caracter\'{i}sticas}
\label{vetores de caracterisctica}
\par Vetor de caracter\'{i}stica...
%%%%%%%%%%%%%%%%%%%%%%%%%%%%%%%%%%%%%%%%%%%%%%%%%%%%%%%%

%%%%%%%%%%%%%%%%%%%%%%%%%%%%%%%%%%%%%%%%%%%%%%%%%%%%%%%%
% SECAO
%%%%%%%%%%%%%%%%%%%%%%%%%%%%%%%%%%%%%%%%%%%%%%%%%%%%%%%%
\section{N\'{i}veis Cr\'{i}ticos de Energia}
\label{niveis criticos de energia}
\par Niveis de energia sao...
%%%%%%%%%%%%%%%%%%%%%%%%%%%%%%%%%%%%%%%%%%%%%%%%%%%%%%%%

%%%%%%%%%%%%%%%%%%%%%%%%%%%%%%%%%%%%%%%%%%%%%%%%%%%%%%%%
% CAPITULO 3
%%%%%%%%%%%%%%%%%%%%%%%%%%%%%%%%%%%%%%%%%%%%%%%%%%%%%%%%
%%%%%%%%%%%%%%%%%%%%%%%%%%%%%%%%%%%%%%%%%%%%%%%%%%%%%%%%
\chapter{Detalhamento do Trabalho Proposto}
\label{cap3}
\thispagestyle{myheadings}
%%%%%%%%%%%%%%%%%%%%%%%%%%%%%%%%%%%%%%%%%%%%%%%%%%%%%%%%
% SECAO
%%%%%%%%%%%%%%%%%%%%%%%%%%%%%%%%%%%%%%%%%%%%%%%%%%%%%%%%
\vspace*{-0.3cm}
\section{Considera\c{c}\~{o}es iniciais}
\par Como primeira etapa para ....
%%%%%%%%%%%%%%%%%%%%%%%%%%%%%%%%%%%%%%%%%%%%%%%%%%%%%%%%
%%%%%%%%%%%%%%%%%%%%%%%%%%%%%%%%%%%%%%%%%%%%%%%%%%%%%%%%
% CAPITULO 4
%%%%%%%%%%%%%%%%%%%%%%%%%%%%%%%%%%%%%%%%%%%%%%%%%%%%%%%%
%%%%%%%%%%%%%%%%%%%%%%%%%%%%%%%%%%%%%%%%%%%%%%%%%%%%%%%%
\chapter{Testes e Resultados}
\label{cap4}
\thispagestyle{myheadings}
bla bla bla...
%%%%%%%%%%%%%%%%%%%%%%%%%%%%%%%%%%%%%%%%%%%%%%%%%%%%%%%%
%%%%%%%%%%%%%%%%%%%%%%%%%%%%%%%%%%%%%%%%%%%%%%%%%%%%%%%%
% CAPITULO 5
%%%%%%%%%%%%%%%%%%%%%%%%%%%%%%%%%%%%%%%%%%%%%%%%%%%%%%%%
%%%%%%%%%%%%%%%%%%%%%%%%%%%%%%%%%%%%%%%%%%%%%%%%%%%%%%%%
\chapter{Conclus\~{o}es e Trabalhos Futuros}
\label{cap5}
\thispagestyle{myheadings}
\par Neste trabalho, ...
%%%%%%%%%%%%%%%%%%%%%%%%%%%%%%%%%%%%%%%%%%%%%%%%%%%%%%%%
%%%%%%%%%%%%%%%%%%%%%%%%%%%%%%%%%%%%%%%%%%%%%%%%%%%%%%%%
% REFERENCIAS
%%%%%%%%%%%%%%%%%%%%%%%%%%%%%%%%%%%%%%%%%%%%%%%%%%%%%%%%
%%%%%%%%%%%%%%%%%%%%%%%%%%%%%%%%%%%%%%%%%%%%%%%%%%%%%%%%
\renewcommand
\bibname{\centering{Refer\^{e}ncias}}
\addcontentsline{toc}{chapter}{Refer\^{e}ncias}
\begin{thebibliography}{00}
\thispagestyle{myheadings}
\bibliographystyle{abnt-num}
\bibitem{PETRY2002}\begin{singlespace}PETRY, A. \textit{Reconhecimento autom\'{a}tico de locutor utilizando medidas de invariantes din\^{a}micas n\~{a}o-lineares}. 2002. 155 p. Tese (Doutorado em Ci\^{e}ncia da Computa\c{c}\~{a}o)-Instituto de Inform\'{a}tica, Universidade Federal do Rio Grande do Sul, Porto Alegre, 2002.\end{singlespace}
\bibitem{forensic_speaker_recognition}\begin{singlespace}CAMPBELL, J. P. et al. Forensic speaker recognition: a need for caution. \textit{IEEE Signal Processing Magazine}, v. 26, n. 2, p. 95-103, 2009. doi:10.1109/msp.2008.931100.\end{singlespace}
\bibitem{abff}\begin{singlespace}ACADEFORD. Dispon\'{i}vel em:$<$http://www.acadeffor.com.br/$>$. Acesso em: 12 ago. 2014.\end{singlespace}
\begin{singlespace}
num\'{e}rico. S\~{a}o Paulo: Pearson Prentice Hall, 2007.\end{singlespace}
\end{thebibliography}
%%%%%%%%%%%%%%%%%%%%%%%%%%%%%%%%%%%%%%%%%%%%%%%%%%%%%%%%
%%%%%%%%%%%%%%%%%%%%%%%%%%%%%%%%%%%%%%%%%%%%%%%%%%%%%%%%
% APENDICE
%%%%%%%%%%%%%%%%%%%%%%%%%%%%%%%%%%%%%%%%%%%%%%%%%%%%%%%%
%%%%%%%%%%%%%%%%%%%%%%%%%%%%%%%%%%%%%%%%%%%%%%%%%%%%%%%%
\thispagestyle{myheadings}
\newpage\ \thispagestyle{myheadings} \newpage
\thispagestyle{myheadings}

\appendix

\renewcommand{\chaptername}{Ap\^{e}ndice I - Gr\'{a}ficos das caracter\'{i}sticas extra\'{i}das}

\addcontentsline{toc}{chapter}{Ap\^{e}ndice I - Gr\'{a}ficos das caracter\'{i}sticas extra\'{i}das}

\noindent

{\huge \vspace*{+79pt} \textbf{Ap\^{e}ndice I - Gr\'{a}ficos das caracter\'{i}sticas}}
{\huge \hspace*{+11pt} \textbf{extra\'{i}das}}

\end{document}