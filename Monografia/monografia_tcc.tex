%%%%%%%%%%%%%%%%%%%%%%%%%%%%%%%%%%%%%%%%%%%%%%%%%%%%%%%%
%%%%%%%%%%%%%%%%%%%%%%%%%%%%%%%%%%%%%%%%%%%%%%%%%%%%%%%%
%%%%%%%%%%%%%%%%%%%%%%%%%%%%%%%%%%%%%%%%%%%%%%%%%%%%%%%%
%%%%%%%%%%%%%%%%%%%%%%%%%%%%%%%%%%%%%%%%%%%%%%%%%%%%%%%%
\documentclass[a4paper,12pt,twoside,openright]{report}
\usepackage[bibjustif]{abntex2cite}
\usepackage[brazil]{babel}
\usepackage{setspace}
\usepackage{amsthm}
\usepackage[figuresright]{rotating}
\usepackage{graphics}
\usepackage{amssymb}
\usepackage{graphicx}
\usepackage{fancybox}
\usepackage{amsmath}
\usepackage{picinpar}
\usepackage{colortbl}
\usepackage{wasysym}
\usepackage{txfonts}
\usepackage{pb-diagram}
\usepackage{relsize}
\usepackage{tikz}
\usepackage{pgfplots}
\usepackage{subfigure}
\usepackage{algorithm}
\usepackage{algorithmic}
\usepackage[hang,small,bf]{caption}
\usepackage[compact]{titlesec}
\usepackage[left=3cm,top=3cm,right=2cm,bottom=3cm]{geometry}%esq e sup 3cm  ; dir e inf 2cm
\usepackage[subfigure]{tocloft}
\usepackage{subfigure}

\renewcommand{\cftfigfont}{Figura }
\renewcommand{\cfttabfont}{Tabela }

\renewcommand{\cftfigaftersnum}{ - }
\renewcommand{\cfttabaftersnum}{ - }

\captionsetup{font=footnotesize,labelsep=endash}

\def\baselinestretch{1.6}
\pagestyle{myheadings}

\begin{document}
\titlespacing{\section}{0cm}{1.5cm}{1.5cm}
\titlespacing{\subsection}{0cm}{1.5cm}{1.5cm}

%%%%%%%%%%%%%%%%%%%%%%%%%%%%%%%%%%%%%%%%%%%%%%%%%%%%%%%%
%%%%%%%%%%%%%%%%%%%%%%%%%%%%%%%%%%%%%%%%%%%%%%%%%%%%%%%%
% CAPA
%%%%%%%%%%%%%%%%%%%%%%%%%%%%%%%%%%%%%%%%%%%%%%%%%%%%%%%%
%%%%%%%%%%%%%%%%%%%%%%%%%%%%%%%%%%%%%%%%%%%%%%%%%%%%%%%%
\setboolean{@twoside}{true}
\title{
	\vspace*{-120pt}
	\hspace*{+55pt}Universidade Estadual Paulista
	\newline
	\hspace*{+45pt}Instituto de Bioci\^{e}ncias, Letras e Ci\^{e}ncias Exatas
	\newline
	\hspace*{+30pt}Departamento de Ci\^{e}ncia da Computa\c{c}\~{a}o e Estat\'{i}stica
	\newline
	\newline
	\newline
	\hspace*{+50pt}Luis Fernando Teixeira Silva
	\newline
	\newline
	\newline
	Um sistema para reconhecimento de comandos falados dependente do locutor
}

\author{{\tiny .}}
\date{\vspace*{+60pt}S\~{a}o Jos\'{e} do Rio Preto - SP \\ 2017}
\maketitle

\newpage\
\thispagestyle{empty}

%%%%%%%%%%%%%%%%%%%%%%%%%%%%%%%%%%%%%%%%%%%%%%%%%%%%%%%%
%%%%%%%%%%%%%%%%%%%%%%%%%%%%%%%%%%%%%%%%%%%%%%%%%%%%%%%%
% PAGINA DE ROSTO
%%%%%%%%%%%%%%%%%%%%%%%%%%%%%%%%%%%%%%%%%%%%%%%%%%%%%%%%
%%%%%%%%%%%%%%%%%%%%%%%%%%%%%%%%%%%%%%%%%%%%%%%%%%%%%%%%
\pagestyle{empty}
\begin{tabular}{p{430pt}}
\pagestyle{empty}
\begin{center}
\pagestyle{empty}
\LARGE{\hspace*{+50pt}Luis Fernando Teixeira Silva\newline \newline \newline \newline \newline Um sistema para reconhecimento de comandos falados dependente do locutor}
\end{center}

\vspace*{+30pt}
\hspace*{+200pt}
\begin{tabular}{p{200pt}}
	\singlespacing{\small Monografia apresentada ao Programa de gradua\c{c}\~{a}o em Ci\^{e}ncia da Computa\c{c}\~{a}o da UNESP para obten\c{c}\~{a}o do t\'{i}tulo de Bacharel. \newline \newline Orientador: Prof. Dr. Rodrigo Capobianco Guido} \\
\end{tabular}
\newline 
\newline
\begin{center}
{\large S\~{a}o Jos\'{e} do Rio Preto - SP \\ 2017}
\end{center}
\end{tabular}

%%%%%%%%%%%%%%%%%%%%%%%%%%%%%%%%%%%%%%%%%%%%%%%%%%%%%%%%
%%%%%%%%%%%%%%%%%%%%%%%%%%%%%%%%%%%%%%%%%%%%%%%%%%%%%%%%
% FICHA CATALOGRAFICA
%%%%%%%%%%%%%%%%%%%%%%%%%%%%%%%%%%%%%%%%%%%%%%%%%%%%%%%%
%%%%%%%%%%%%%%%%%%%%%%%%%%%%%%%%%%%%%%%%%%%%%%%%%%%%%%%%


\begin{center}
	\vspace*{+260pt}
	{\footnotesize \hspace*{-81pt}Ficha catalogr\'{a}fica elaborada pelo Servi\c{c}o de Biblioteca do IBILCE/UNESP}
\end{center}

\vspace*{+5pt}

\begin{tabular}{|p{12.5cm}|}
	\hline
	{\singlespacing
		\vspace*{-20pt}
		\hspace*{+40pt} {\small Luis Fernando Teixeira Silva} \newline
		\hspace*{+40pt} {\small titulo} \newline
		\hspace*{+20pt} {\small titulo} \newline
		\hspace*{+20pt} {\small titulo. / fulano de tal; orientador} \newline
		\hspace*{+20pt} {\small Rodrigo Capobianco Guido. S\~{a}o Jos\'{e} do Rio Preto, 2017.} \newline
		\hspace*{+40pt} {\small xxx p.} \newline
		
		\hspace*{+40pt} {\small Monografia (TCC} \newline
		\hspace*{+20pt} {\small TCC} \newline
		\hspace*{+20pt} {\small TCC, 2017.} \newline \newline
		
		\hspace*{+40pt} {\small 1. Processamento de sinais. 2. Reconhecimento de locutor. 3. Ac\'{u}s-} \newline
		\hspace*{+20pt} {\small tica. 4. Energia. 5. Escala \textit{Bark}. I. Capobianco Guido, Rodrigo, orient.} \newline
		\hspace*{+20pt} {\small II. T\'{i}tulo.} \newline \newline
	}
	\\
	\hline
\end{tabular}

%%%%%%%%%%%%%%%%%%%%%%%%%%%%%%%%%%%%%%%%%%%%%%%%%%%%%%%%
%%%%%%%%%%%%%%%%%%%%%%%%%%%%%%%%%%%%%%%%%%%%%%%%%%%%%%%%
% FOLHA DE APROVACAO: MANTER EM BRANCO
%%%%%%%%%%%%%%%%%%%%%%%%%%%%%%%%%%%%%%%%%%%%%%%%%%%%%%%%
%%%%%%%%%%%%%%%%%%%%%%%%%%%%%%%%%%%%%%%%%%%%%%%%%%%%%%%%
\thispagestyle{empty}
\newpage\ \thispagestyle{empty} \newpage
\thispagestyle{empty}
\newpage\ \thispagestyle{empty} \newpage
\thispagestyle{empty}

%%%%%%%%%%%%%%%%%%%%%%%%%%%%%%%%%%%%%%%%%%%%%%%%%%%%%%%%
%%%%%%%%%%%%%%%%%%%%%%%%%%%%%%%%%%%%%%%%%%%%%%%%%%%%%%%%
% DEDICATORIA
%%%%%%%%%%%%%%%%%%%%%%%%%%%%%%%%%%%%%%%%%%%%%%%%%%%%%%%%
%%%%%%%%%%%%%%%%%%%%%%%%%%%%%%%%%%%%%%%%%%%%%%%%%%%%%%%%
\newpage
\thispagestyle{empty}
\vspace*{+480pt}
\hspace*{+160pt}
\begin{tabular}{p{200pt}}
	{\small Dedico este trabaho a todos os meus familiares, em especial aos meus pais, Nilda, Luis Carlos e a minha irm\~{a} Ana Beatriz.\newline Dedico tamb\'{e}m esse trabalho para a minha namorada Cristiana Luiza.}
\end{tabular}

\newpage\ \thispagestyle{empty} \newpage

%%%%%%%%%%%%%%%%%%%%%%%%%%%%%%%%%%%%%%%%%%%%%%%%%%%%%%%%
%%%%%%%%%%%%%%%%%%%%%%%%%%%%%%%%%%%%%%%%%%%%%%%%%%%%%%%%
% AGRADECIMENTOS
%%%%%%%%%%%%%%%%%%%%%%%%%%%%%%%%%%%%%%%%%%%%%%%%%%%%%%%%
%%%%%%%%%%%%%%%%%%%%%%%%%%%%%%%%%%%%%%%%%%%%%%%%%%%%%%%%
\newpage
\thispagestyle{empty}
\begin{center}{\LARGE \textbf{Agradecimentos}}\end{center}
\vspace{+60pt}
\par Primeiramente, gostaria de agradecer meus pais e minha madrinha, pois sem o apoio deles eu nunca teria conseguido ter acesso a um ensino de qualidade que o cursinho alternativo me proporcionou durante todo o ano de 2012. Foi gra{\c c}as a essas 3 pessoas que pude ingressar nessa linda universidade.

\par Gostaria de agradecer tamb\'{e}m a minha irm\~{a} que nos momentos mais dif\'{i}cies da minha gradua{\c c}\~{a}o me deu for{\c c}as para continuar em frente e concluir minha forma{\c c}\~{a}o de bacharel em ci\^{e}ncia da computa{\c c}\~{a}o. Agrade{\c c}o tamb\'{e}m a todos os meus familiares que me apoiaram ao longos dessa jornada de 5 anos.

\par E tamb\'{e}m deixo um agradecimento especial a meus dois grandes amigos Jo\~{a}o Cesar Granville e Luiz Gustavo Caobianco que tornaram esses anos na universidade mais felizes. Agrade{\c c}o tamb\'{e}m minha namorada por ter me auxilidado nesses dois \'{u}ltimos anos de universidade e por me dar for{\c c}as a concluir o curso nessa etapa final.

\newpage\ \thispagestyle{empty} \newpage

%%%%%%%%%%%%%%%%%%%%%%%%%%%%%%%%%%%%%%%%%%%%%%%%%%%%%%%%
%%%%%%%%%%%%%%%%%%%%%%%%%%%%%%%%%%%%%%%%%%%%%%%%%%%%%%%%
% PENSAMENTO
%%%%%%%%%%%%%%%%%%%%%%%%%%%%%%%%%%%%%%%%%%%%%%%%%%%%%%%%
%%%%%%%%%%%%%%%%%%%%%%%%%%%%%%%%%%%%%%%%%%%%%%%%%%%%%%%%
\newpage
\vspace*{+480pt}
\thispagestyle{empty}
\begin{flushright}
\textit{``No fim tudo d\'{a} certo, e se n\~{a}o deu certo \'{e} porque ainda n\~{a}o chegou ao fim.''}
\\
\textbf{Fernando Sabino}
\end{flushright}

\newpage\ \thispagestyle{empty} \newpage

%%%%%%%%%%%%%%%%%%%%%%%%%%%%%%%%%%%%%%%%%%%%%%%%%%%%%%%%
%%%%%%%%%%%%%%%%%%%%%%%%%%%%%%%%%%%%%%%%%%%%%%%%%%%%%%%%
% RESUMO E ABSTRACT
%%%%%%%%%%%%%%%%%%%%%%%%%%%%%%%%%%%%%%%%%%%%%%%%%%%%%%%%
%%%%%%%%%%%%%%%%%%%%%%%%%%%%%%%%%%%%%%%%%%%%%%%%%%%%%%%%
\newpage
\thispagestyle{empty}
\noindent
\begin{center}\textbf{\huge{Resumo}}\end{center} 
\vspace*{+60pt}
\begin{singlespace}
TAL, F. \textit{titulo}. 2016. xxxp. TCC UNESP 2016.
\end{singlespace}
\vspace*{+20pt}
Atualmente, ....
\\
\\
Palavras-chave: Processamento de sinais. Reconhecimento de locutor. Ac\'{u}stica. Escala \textit{Bark}.

\newpage\ \thispagestyle{empty} \newpage

\newpage
\thispagestyle{empty}
\noindent
\begin{center}\textbf{\huge{Abstract}}\end{center} 
\vspace*{+60pt}
\begin{singlespace}
TAL, F. \textit{titulo}. 2016. xxxp. TCC UNESP 2017.
\end{singlespace}
\vspace*{+20pt}
Nowadays, ...
\\
\\
Keywords: Signal processing. Speaker recognition. Acoustics. Bark scale.

\newpage\ \thispagestyle{empty} \newpage

%%%%%%%%%%%%%%%%%%%%%%%%%%%%%%%%%%%%%%%%%%%%%%%%%%%%%%%%
%%%%%%%%%%%%%%%%%%%%%%%%%%%%%%%%%%%%%%%%%%%%%%%%%%%%%%%%
% INDICES
%%%%%%%%%%%%%%%%%%%%%%%%%%%%%%%%%%%%%%%%%%%%%%%%%%%%%%%%
%%%%%%%%%%%%%%%%%%%%%%%%%%%%%%%%%%%%%%%%%%%%%%%%%%%%%%%%
%\pagestyle{empty}
\renewcommand
\listfigurename{\hspace*{+120pt} Lista de Figuras \thispagestyle{empty}}
\pagestyle{empty}
\listoffigures
%\pagestyle{empty}
\newpage\ \thispagestyle{empty} \newpage
%\pagestyle{empty}
\renewcommand
\listtablename{\hspace*{+120pt} Lista de Tabelas \thispagestyle{empty}}
%\pagestyle{empty}
\listoftables
%\pagestyle{empty}

\newpage\ \thispagestyle{empty} \newpage

%%%%%%%%%%%%%%%%%%%%%%%%%%%%%%%%%%%%%%%%%%%%%%%%%%%%%%%%
%%%%%%%%%%%%%%%%%%%%%%%%%%%%%%%%%%%%%%%%%%%%%%%%%%%%%%%%
% LISTA DE ABREVIACOES
%%%%%%%%%%%%%%%%%%%%%%%%%%%%%%%%%%%%%%%%%%%%%%%%%%%%%%%%
%%%%%%%%%%%%%%%%%%%%%%%%%%%%%%%%%%%%%%%%%%%%%%%%%%%%%%%%
\newpage
\thispagestyle{empty}
\vspace*{+62pt}
\begin{center}{\huge \textbf{Lista de Abreviaturas}}\end{center}
\vspace*{+20pt}
\begin{tabular}{l l}
\textbf{WAVE}&\textit{Waveform Audio File Format}
\\
\textbf{PCM}&\textit{Pulse-Code Modulation}
\\
\textbf{IBM}&\textit{International Business Machines}
\\
\textbf{RIFF}&\textit{Resource Interchange File Format}
\\
\textbf{fmt}&\textit{format}
\\
\textbf{IEEE}&\textit{Institute of Electrical and Eletronic Engineers}

\end{tabular}

%%%%%%%%%%%%%%%%%%%%%%%%%%%%%%%%%%%%%%%%%%%%%%%%%%%%%%%%
%%%%%%%%%%%%%%%%%%%%%%%%%%%%%%%%%%%%%%%%%%%%%%%%%%%%%%%%
% SUMARIO
%%%%%%%%%%%%%%%%%%%%%%%%%%%%%%%%%%%%%%%%%%%%%%%%%%%%%%%%
%%%%%%%%%%%%%%%%%%%%%%%%%%%%%%%%%%%%%%%%%%%%%%%%%%%%%%%%
\newpage\ \thispagestyle{empty} \newpage
%\pagestyle{empty}
\renewcommand
\contentsname{\hspace*{+160pt} Sum\'{a}rio \thispagestyle{empty}}
%\pagestyle{empty}
\tableofcontents
%\pagestyle{empty}

%%%%%%%%%%%%%%%%%%%%%%%%%%%%%%%%%%%%%%%%%%%%%%%%%%%%%%%%
%%%%%%%%%%%%%%%%%%%%%%%%%%%%%%%%%%%%%%%%%%%%%%%%%%%%%%%%
% CAPITULO 1
%%%%%%%%%%%%%%%%%%%%%%%%%%%%%%%%%%%%%%%%%%%%%%%%%%%%%%%%
%%%%%%%%%%%%%%%%%%%%%%%%%%%%%%%%%%%%%%%%%%%%%%%%%%%%%%%%
\setboolean{@twoside}{true}
\chapter{Introdu\c{c}\~{a}o}
\thispagestyle{myheadings}
\pagestyle{myheadings}
%%%%%%%%%%%%%%%%%%%%%%%%%%%%%%%%%%%%%%%%%%%%%%%%%%%%%%%%
% SECAO
%%%%%%%%%%%%%%%%%%%%%%%%%%%%%%%%%%%%%%%%%%%%%%%%%%%%%%%%
\section{Introdu\c{c}\~{a}o}
\label{cap1}
\par Petry \cite{PETRY2002} define o reconhecimento de locutores como...
%%%%%%%%%%%%%%%%%%%%%%%%%%%%%%%%%%%%%%%%%%%%%%%%%%%%%%%%
% SECAO
%%%%%%%%%%%%%%%%%%%%%%%%%%%%%%%%%%%%%%%%%%%%%%%%%%%%%%%%
\section{Objetivo}
\par Este trabalho tem como objetivo implementar um algoritmo computacional desenvolvido em C/C++ para reconhecer comandos falados de modo \textit{off-line} com locutor pr\'{e}definido, ou seja \textit{speaker-dependent}. Esses comandos foram previamente gravados em arquivos no formato \textit{WAVE} de 16 \textit{bits} PCM.
%%%%%%%%%%%%%%%%%%%%%%%%%%%%%%%%%%%%%%%%%%%%%%%%%%%%%%%%

% SECAO
%%%%%%%%%%%%%%%%%%%%%%%%%%%%%%%%%%%%%%%%%%%%%%%%%%%%%%%%
\section{justificativa}
\par justificar e oferecer razao suficiente para a construcao desse trabalho. Responde a pergunta do que pq fazer o trabalho, procurando os antecedentes do problema e a relevancia do assunto, argumentando sobre sua importancia pratico/teorica, colocando as possiveis contribuicoes esperadas
%%%%%%%%%%%%%%%%%%%%%%%%%%%%%%%%%%%%%%%%%%%%%%%%%%%%%%%%

% SECAO
%%%%%%%%%%%%%%%%%%%%%%%%%%%%%%%%%%%%%%%%%%%%%%%%%%%%%%%%
\section{Motiva\c{c}\~{a}o}
\par Em 2008, \'{e} lan{\c c}ado o primeiro filme do Homem de Ferro, centrado no personagem Tony Stark. O filme al\'{e}m de despertar o interesse nas hist\'{o}rias em quadrinhos da Marvel, tamb\'{e}m prende a aten{\c c}\~{a}o dos ci\^{e}ntistas da computa{\c c}\~{a}o, uma vez que o personagem principal interage com uma intelig\^{e}ncia artificial - Jarvis - que podia controlar a casa e a armadura do h\'{e}roi. J\'{a} no ano de 2011, foi lan{\c c}ado a primeira vers\'{a}o do assistente pessoal do iOS, conhecido como Siri, que permite que o usu\'{a}rio execute determinadas fun{\c c}\~{o}es do \textit{smartphone} utilizando comandos falados. E mais recente, o criador do Facebook, decidiu no ano 2016, implementar seu pr\'{o}prio assistente pessoal, para auxiliar nas tarefas dom\'{e}sticas.
\par FALAR SOBRE ALGUM CONTEXTO CIENTIFICO TAMBEM.
\par E \'{e} a partir desse contexto que surgiu a inspira{\c c}\~{a}o para o desenvolvimento desse trabalho. Inicialmente, o projeto tinha como proposta de conseguir auxiliar em determinadas fun{\c c}\~{o}es de uma resid\^{e}ncia, al\'{e}m de executar comandos b\'{a}sico em um computador. Por\'{e}m, o projeto necessitou de cortes em seu escopo para que se tornasse fact\'{i}vel sua elabora{\c c}\~{a}o como forma de trabalho de conclus\~{a}o de curso.
\par Assim, este projeto tem como objetivo iniciar o desenvolvimento de um futuro assitente pessoal, atrav\'{e}s da elabora{\c c}\~{a}o de um sistema para reconhecimento de comandos falados \textit{speaker-dependent}. 
%%%%%%%%%%%%%%%%%%%%%%%%%%%%%%%%%%%%%%%%%%%%%%%%%%%%%%%%

% SECAO
%%%%%%%%%%%%%%%%%%%%%%%%%%%%%%%%%%%%%%%%%%%%%%%%%%%%%%%%
\section{Metodologia}
\par Para a elabora\c{c}\~{a}o deste projeto foi determinado os seguintes 11 comandos:
\begin{itemize}
	\item{}Bom dia, Logan;
	\item{}Bom noite, Logan;
	\item{}Oi, Logan;
	\item{}Como est\'{a} o tempo hoje?;
	\item{}vai chover?;
	\item{}Abrir calculadora;
	\item{}Ver not\'{i}cias;
	\item{}Pesquisar;
	\item{}Alarme;
	\item{}Calend\'{a}rio;
	\item{}Sair;	
\end{itemize}
sendo que posteriormente foi realizada a grava\c{c}\~{a}o de 10 \'{a}udios para cada um dos 11 comandos referidos, totalizando 110 arquivos de \'{a}udio no formato MPEG-4.Tais arquivos foram convertidos para o formato \textit{WAVE} de 16 \textit{bits} PCM usando o programa \textit{Audacity}. Vale ressaltar que todos os \'{a}udios foram gravados em um ambiente que proporciona-se um certo grau de isolamento sonoro, para assim se obter um som com menos ru\'{i}do.
\par A partir dessa etapa inicial foi feita a extra\c{c}\~{a}o dos dados brutos contidos nos arquivos \textit{WAVE}. Para isso foi utilizada uma biblioteca fornecida pelo Prof.Dr.Rodrigo Capobianco Guido do Departamento de Ci\^{e}ncia da Comput{\c c}\~{a}o e Estat\'{i}stica (DCCE), IBILCE/Unesp. Tal biblioteca, escrita em C/C++, tem a fun\c{c}\~{a}o de separar o cabe{\c c}alho dos arquivos \textit{WAVE}. A partir desse ponto, a biblioteca foi modificada para extrair os dados brutos e guardar as amplitudes dos sinais em arquivos de texto. Foi realizada a automa{\c c}\~{a}o de todo o processo que exigia interven{\c c}\~{a}o humana para a execu{\c c}\~{a}o do algoritmo, como a passagem  de \'{a}udios como par\~{a}metro a cada nova execu{\c c}\~{a}o, cria{\c c}\~{a}o de arquivos para extra{\c c}\~{a}o dos valores das amplitudes dos sinais, entre outros. Toda essa automatiza{\c c}\~{a}o foi implementada com a utiliza{\c c}\~{a}o de \textit{scripts} escritos na linguagem \textit{Shell script}.
\par Posteriormente a etapa de extra{\c c}\~{a}o das amplitudes dos sinais digitalizados, foi realizada a extra{\c c}\~{a}o das caracter\'{i}sticas dos \'{a}udios analisados. Essa etapa consiste na utiliza{\c c}\~{a}o do m\'{e}todo A3 - descrito com maiores detalhes no \ref{cap2} - para obter assim, vetores de caracter\'{i}sticas. Tal processo \'{e} essencial no presente trabalho, pois os valores obtidos no processo de extra{\c c}\~{a}o s\~{a}o vari\'{a}veis e demasiadamente grandes, sendo que o classificador exige valores menores e com tamanho fixo.
%%%%%%%%%%%%%%%%%%%%%%%%%%%%%%%%%%%%%%%%%%%%%%%%%%%%%%%%
% SECAO
%%%%%%%%%%%%%%%%%%%%%%%%%%%%%%%%%%%%%%%%%%%%%%%%%%%%%%%%
\section{Exequibilidade}
\par Exequibilidade...
%%%%%%%%%%%%%%%%%%%%%%%%%%%%%%%%%%%%%%%%%%%%%%%%%%%%%%%%

% SECAO
%%%%%%%%%%%%%%%%%%%%%%%%%%%%%%%%%%%%%%%%%%%%%%%%%%%%%%%%
\section{Organiza\c{c}\~{a}o do trabalho}
\par A monografia est\'{a} organizada a partir deste cap\'{i}tulo da seguinte forma:

\begin{itemize}
\item{}No Cap\'{i}tulo \ref{cap2} apresenta-se uma s\'{e}rie de trabalhos realizados na \'{a}rea \textit{speaker-dependent} tanto a n\'{i}vel local quanto a internacional, exaltando a relev\^{a}ncia da \'{a}rea no \^{a}mbito acad\^{e}mico. Al\'{e}m disso, \'{e} apresentado tamb\'{e}m neste cap\'{i}tulo toda a fundamenta{\c c}\~{a}o te\'{o}rica do trabalho, definindo e exemplificando os principais conceitos utilizados na elabora{\c c}\~{a}o do deste projeto.
\end{itemize}

\begin{itemize}
	\item{}No Cap\'{i}tulo \ref{cap3} apresenta-se uma breve descri{\c c}\~{a}o do estado do atual do trabalho e tamb\'{e}m um cronograma para finaliza{\c c}\~{a}o.
\end{itemize}
%%%%%%%%%%%%%%%%%%%%%%%%%%%%%%%%%%%%%%%%%%%%%%%%%%%%%%%%
%%%%%%%%%%%%%%%%%%%%%%%%%%%%%%%%%%%%%%%%%%%%%%%%%%%%%%%%
% CAPITULO 2
%%%%%%%%%%%%%%%%%%%%%%%%%%%%%%%%%%%%%%%%%%%%%%%%%%%%%%%%
%%%%%%%%%%%%%%%%%%%%%%%%%%%%%%%%%%%%%%%%%%%%%%%%%%%%%%%%
\chapter{Revis\~{a}o Bibliogr\'{a}fica}
\label{cap2}
\thispagestyle{myheadings}
%%%%%%%%%%%%%%%%%%%%%%%%%%%%%%%%%%%%%%%%%%%%%%%%%%%%%%%%
% SECAO
%%%%%%%%%%%%%%%%%%%%%%%%%%%%%%%%%%%%%%%%%%%%%%%%%%%%%%%%
\section{Fundamenta\c{c}\~{a}o da Verifica\c{c}\~{a}o de Locutores}
\label{secao_reconhecimento}
\par Reconhecimento de locutores \cite{BEIGI2011}...
%%%%%%%%%%%%%%%%%%%%%%%%%%%%%%%%%%%%%%%%%%%%%%%%%%%%%%%%
% SECAO
%%%%%%%%%%%%%%%%%%%%%%%%%%%%%%%%%%%%%%%%%%%%%%%%%%%%%%%%
\section{Arquivos Ac\'{u}sticos no Formato \textit{WAVE}}
\label{secao_formato wave}

\par\textit{Waveform audio file format} \'{e} a abrevia\c{c}\~{a}o de \textit{WAVE} ou simplesmente \textit{WAV}, que \'{e} um tipo de formato de arquivo de \'{a}udio que foi desenvolvido pela \textit{Microsoft} em conjunto com a IBM. O formato \textit{WAVE} \'{e} amplamente utilizado em uma variedade de trabalhos, sejam eles cient\'{i}ficos ou profissionais, visto que o formato permite uma fiel representa\c{c}\~{a}o dos dados digitalizados, uma vez que os dados digitalizados podem ser armazenados sem sofrer obrigatoriamente um processo de compress\~{a}o, o que evita perdas. Por\'{e}m, devido a essa caracter\'{i}stica o \textit{WAV} ocupa muito mais espa\c{c}o que os demais formatos de arquivos de \'{a}udios. 

\par A Tabela 2.1 mostra a estrutura de um arquivo \textit{WAVE}. Basicamente o arquivo \'{e} divido em 2 grandes blocos, sendo o primeiro bloco um cabe{\c c}alho RIFF e o segundo bloco \'{e} divido em dois sub-blocos, sendo um com informa{\c c}\~{o}es referentes ao formato \textit{WAVE} e o outro com os dados do \'{a}udios.

\par Vale ressaltar que os valores mais comuns para cada amostra de um  arquivo \textit{WAVE} pode ser 8 \textit{bits} ou 16 \textit{bits}. Um \'{a}udio de 8 \textit{bits} siginifica que o valor da amplitude do sinal, de cada amostra, pode ser representado por 256 valores, sendo 127 positivos e 128 negativos. J\'{a} para um arquivo 16 \textit{bits} a amplitude do sinal pode ser representado por 65536 valores, com 32757 positivos e 32768 negativos. Para \textit{WAVE} de 16 \textit{bits} \'{e} utilizada a codifica{\c c}\~{a}o de complemento de 2 para representar o valor da amplitude do sinal. Assim, o valor do \textit{bit} mais significativo representa se o sinal \'{e} negaivo ou positivo.

\par Nesse trabalho foi utilizado o formato \textit{WAV} de 16 \textit{bits} PCM (\textit{Pulse-code Modulation}) que n\~{a}o utiliza compress\~{a}o, para se obter assim uma melhor qualidade na elabora\c{c}\~{a}o deste projeto final. Foi fornecida uma biblioteca escrita em C/C++ pelo orientador para isolar o primeiro bloco referente ao cabe{\c c}alho RIFF e o sub-bloco de formato \textit{WAV} dos dados brutos, que cont\'{e}m as amplitudes dos sinais de voz digitalizados.

\vspace*{+10pt}

\begin{table}[h] 

	\caption{Estrutura de um arquivo \textit{WAVE}}
	\begin{tabular}{c|c|c|c}
		\textbf{Classe} & \textbf{Posi\c{c}\~{a}o \textit{(bytes)}} & \textbf{Tamanho \textit{(bytes)}} & \textbf{Descri\c{c}\~{a}o}\\
		\hline
		Cabe\c{c}alho & 0 & 4 & Apresenta o identificador do cabe\c{c}alho - "RIFF".\\
		Cabe\c{c}alho & 4 & 4 & Tamanho do arquivo sem o identificado do cabe\c{c}alho.\\
		Cabe\c{c}alho & 8 & 4 & Mostra o identificador \textit{WAVE}.\\
		\hline
		Formato & 12 & 4 & Mostra o identificador do segundo bloco - "fmt".\\
		Formato & 16 & 4 & Tamanho do bloco sem o identificador.\\
		Formato & 20 & 2 & Mostra se o arquivo \'{e} do tipo PCM ou se tem alguma \\ & & & compress\~{a}o.\\
		Formato & 22 & 2 & Mostra a quantidade de canais.\\
		Formato & 26 & 4 & Apresenta o valor da taxa de amostragem.\\
		Formato & 30 & 4 & Apresenta a taxa de \textit{bytes}.\\
		Formato & 32 & 2 & Demostra a quantidade de \textit{bytes} para uma amostra.\\
		Formato & 34 & 2 & Demostra a quantidade de \textit{bits} para cada amostra.\\
		\hline
		Dados & 36 & 4 & Apresenta o identificador do terceiro bloco - "\textit{data}".\\
		Dados & 40 & 4 & Mostra o tamanho do bloco sem o identificador.\\
		Dados & 44 & 4 & Demostra os dados reais da m\'{u}sica.
		
	\end{tabular}
\end{table}     



%%%%%%%%%%%%%%%%%%%%%%%%%%%%%%%%%%%%%%%%%%%%%%%%%%%%%%%%
% SECAO
%%%%%%%%%%%%%%%%%%%%%%%%%%%%%%%%%%%%%%%%%%%%%%%%%%%%%%%%
\section{Energia}
\label{energia}
\par A defini{\c c}\~{a}o de energia est\'{a} relacionada ao conceito de conseguir realizar trabalho. Neste projeto, ser\'{a} considerada energia a capacidade das estruturas voc\'{a}licas e dos pulm\~{o}es de produzir um sinal ac\'{u}stico. A equa{\c c}\~{a}o 2.1 defini a energia total E(s[.]), de um dado sinal de \'{a}udio digitalizado s[.], de tamanho M.

\begin{equation}
	E(s[.])=\sum_{i = 0}^{M-1} (Si)^2
\end{equation}

\par Para realizar a captura das caracter\'{i}ticas dos \'{a}udios foi utilizado o m\'{e}todo A3, que foi desenvolvido pelo pr\'{o}prio orientador em \cite{Guido_tutorial}. Para realizar a extra{\c c}\~{a}o das caracter\'{i}sticas, tal m\'{e}todo se baseia no em determinar tamanhos ou \'{a}reas proporcionais para atingir n\'{i}veis predefinidos da energia do sinal que se encontra em an\'{a}lise. A3 \'{e} ideal para avaliar os n\'{i}veis de energia de um sinal de voz digitalizado que foi gerado por um agente.
\par Vale ressaltar que A3 defini um n\'{i}vel cr\'{i}tico de energia que varia de 0 a 100{\%}. Em um sinal de \'{a}udio, A3 extrai um vetor de caracter\'{i}sticas dividindo o \'{a}udio em partes proporcionais ao valor definido pelo n\'{i}vel cr\'{i}tico - par\^{a}metro C - de modo que, cria uma janela de tamanho C{\%} e extrai a caracter\'{i}stica dessa faixa de valores do sinal digitalizado, conforme \'{e} mostrado na equa{\c c}\~{a}o 2.2, onde $\epsilon$ \'{e} uma caracter\'{i}stica extraida, E(W0[.]) a energia obtida do in\'{i}cio do sinal at\'{e} o ponto final da janela definido por C, E(Wk[.]) \'{e} a energia total do sinal. Tal processo \'{e} repetido at\'{e} que se atinja o montante total de energia do \'{a}udio, de modo que, defini-se assim um vetor de caracter\'{i}sticas. 

\begin{equation}
	\epsilon = (E(W0[.]) / (\sum_{k = 0}^{T-1}E(Wk[.])))
\end{equation}  



%%%%%%%%%%%%%%%%%%%%%%%%%%%%%%%%%%%%%%%%%%%%%%%%%%%%%%%%
% SECAO
%%%%%%%%%%%%%%%%%%%%%%%%%%%%%%%%%%%%%%%%%%%%%%%%%%%%%%%%
\section{Vetores de Caracter\'{i}sticas}
\label{vetores de caracterisctica}
\par Neste trabalho, vetores de caracter\'{i}sticas s\~{a}o vetores com tamanho fixo que armazenam valores 
%%%%%%%%%%%%%%%%%%%%%%%%%%%%%%%%%%%%%%%%%%%%%%%%%%%%%%%%

%%%%%%%%%%%%%%%%%%%%%%%%%%%%%%%%%%%%%%%%%%%%%%%%%%%%%%%%
% SECAO
%%%%%%%%%%%%%%%%%%%%%%%%%%%%%%%%%%%%%%%%%%%%%%%%%%%%%%%%
\section{N\'{i}veis Cr\'{i}ticos de Energia}
\label{niveis criticos de energia}
\par Niveis de energia sao...
%%%%%%%%%%%%%%%%%%%%%%%%%%%%%%%%%%%%%%%%%%%%%%%%%%%%%%%%

%%%%%%%%%%%%%%%%%%%%%%%%%%%%%%%%%%%%%%%%%%%%%%%%%%%%%%%%
% CAPITULO 3
%%%%%%%%%%%%%%%%%%%%%%%%%%%%%%%%%%%%%%%%%%%%%%%%%%%%%%%%
%%%%%%%%%%%%%%%%%%%%%%%%%%%%%%%%%%%%%%%%%%%%%%%%%%%%%%%%
\chapter{Detalhamento do Trabalho Proposto}
\label{cap3}
\thispagestyle{myheadings}
%%%%%%%%%%%%%%%%%%%%%%%%%%%%%%%%%%%%%%%%%%%%%%%%%%%%%%%%
% SECAO
%%%%%%%%%%%%%%%%%%%%%%%%%%%%%%%%%%%%%%%%%%%%%%%%%%%%%%%%
\vspace*{-0.3cm}
\section{Coleta e elabora{\c c}\~{a}o do banco de \'{a}udios}
\par Inicialmente, foram definidos os 11 comandos que ser\~{a}o reconhecidos pelo sistema, conforme j\'{a} mencionado. Os comandos foram gravados 10 vezes em diferentes dias e hor\'{a}rios, para obter se assim, uma melhor veracidade e fidelidade \`{a} voz do locutor, pois o mesmo pode sofrer altera{\c c}\~{o}es significativas com base na varia{\c c}\~{a}o do seu humor ou estado f\'{i}sico. Como tamb\'{e}m j\'{a} mencionado, a grava{\c c}\~{a}o dos arquivos de \'{a}udio foi realizada em um ambiente fechado, para diminuir assim, a probabilidade de ru\'{i}dos nos sinais. Todos os arquivos foram gravados no formato MPEG-4, que \'{e} o formato dispon\'{i}vel no \textit{software} de grava{\c c}\~{a}o de \'{a}udio do \textit{Windows} 10, e posteriormente foram convertidos para o formato \textit{WAVE} de 16 \textit{bits} PCM, com o aux\'{i}lio do editor de \'{a}udio \textit{Audacity}. Ao total, foram gravados e convertidos 110 arquivos de \'{a}udio.
%%%%%%%%%%%%%%%%%%%%%%%%%%%%%%%%%%%%%%%%%%%%%%%%%%%%%%%%
%%%%%%%%%%%%%%%%%%%%%%%%%%%%%%%%%%%%%%%%%%%%%%%%%%%%%%%%
% CAPITULO 4
%%%%%%%%%%%%%%%%%%%%%%%%%%%%%%%%%%%%%%%%%%%%%%%%%%%%%%%%
%%%%%%%%%%%%%%%%%%%%%%%%%%%%%%%%%%%%%%%%%%%%%%%%%%%%%%%%
\chapter{Testes e Resultados}
\label{cap4}
\thispagestyle{myheadings}
bla bla bla...
%%%%%%%%%%%%%%%%%%%%%%%%%%%%%%%%%%%%%%%%%%%%%%%%%%%%%%%%
%%%%%%%%%%%%%%%%%%%%%%%%%%%%%%%%%%%%%%%%%%%%%%%%%%%%%%%%
% CAPITULO 5
%%%%%%%%%%%%%%%%%%%%%%%%%%%%%%%%%%%%%%%%%%%%%%%%%%%%%%%%
%%%%%%%%%%%%%%%%%%%%%%%%%%%%%%%%%%%%%%%%%%%%%%%%%%%%%%%%
\chapter{Conclus\~{o}es e Trabalhos Futuros}
\label{cap5}
\thispagestyle{myheadings}
\par Neste trabalho, ...
%%%%%%%%%%%%%%%%%%%%%%%%%%%%%%%%%%%%%%%%%%%%%%%%%%%%%%%%
%%%%%%%%%%%%%%%%%%%%%%%%%%%%%%%%%%%%%%%%%%%%%%%%%%%%%%%%
% REFERENCIAS
%%%%%%%%%%%%%%%%%%%%%%%%%%%%%%%%%%%%%%%%%%%%%%%%%%%%%%%%
%%%%%%%%%%%%%%%%%%%%%%%%%%%%%%%%%%%%%%%%%%%%%%%%%%%%%%%%
\renewcommand
\bibname{\centering{Refer\^{e}ncias}}
\addcontentsline{toc}{chapter}{Refer\^{e}ncias}
\begin{thebibliography}{00}
\thispagestyle{myheadings}
\bibliographystyle{abnt-num}
\bibitem{PETRY2002}\begin{singlespace}PETRY, A. \textit{Reconhecimento autom\'{a}tico de locutor utilizando medidas de invariantes din\^{a}micas n\~{a}o-lineares}. 2002. 155 p. Tese (Doutorado em Ci\^{e}ncia da Computa\c{c}\~{a}o)-Instituto de Inform\'{a}tica, Universidade Federal do Rio Grande do Sul, Porto Alegre, 2002.\end{singlespace}
\bibitem{forensic_speaker_recognition}\begin{singlespace}CAMPBELL, J. P. et al. Forensic speaker recognition: a need for caution. \textit{IEEE Signal Processing Magazine}, v. 26, n. 2, p. 95-103, 2009. doi:10.1109/msp.2008.931100.\end{singlespace}
\bibitem{abff}\begin{singlespace}ACADEFORD. Dispon\'{i}vel em:$<$http://www.acadeffor.com.br/$>$. Acesso em: 12 ago. 2014.\end{singlespace}
\begin{singlespace}
num\'{e}rico. S\~{a}o Paulo: Pearson Prentice Hall, 2007.\end{singlespace}
\bibitem{Guido_tutorial}
	\begin{singlespace}
		Rodrigo Capobianco Guido. A tutorial on signal energy and its applications.
	\end{singlespace}
\end{thebibliography}
%%%%%%%%%%%%%%%%%%%%%%%%%%%%%%%%%%%%%%%%%%%%%%%%%%%%%%%%
%%%%%%%%%%%%%%%%%%%%%%%%%%%%%%%%%%%%%%%%%%%%%%%%%%%%%%%%
% APENDICE
%%%%%%%%%%%%%%%%%%%%%%%%%%%%%%%%%%%%%%%%%%%%%%%%%%%%%%%%
%%%%%%%%%%%%%%%%%%%%%%%%%%%%%%%%%%%%%%%%%%%%%%%%%%%%%%%%
\thispagestyle{myheadings}
\newpage\ \thispagestyle{myheadings} \newpage
\thispagestyle{myheadings}

\appendix

\renewcommand{\chaptername}{Ap\^{e}ndice I - Gr\'{a}ficos das caracter\'{i}sticas extra\'{i}das}

\addcontentsline{toc}{chapter}{Ap\^{e}ndice I - Gr\'{a}ficos das caracter\'{i}sticas extra\'{i}das}

\noindent

{\huge \vspace*{+79pt} \textbf{Ap\^{e}ndice I - Gr\'{a}ficos das caracter\'{i}sticas}}
{\huge \hspace*{+11pt} \textbf{extra\'{i}das}}

\end{document}